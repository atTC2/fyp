\documentclass[a4paper]{report}

\usepackage[english]{babel}
\usepackage[utf8]{inputenc}
\usepackage{amsmath}
\usepackage{graphicx}
\usepackage{float}
\usepackage[margin=1.in]{geometry}
\usepackage{tocloft}
\usepackage{harvard}
\usepackage{appendix}
\usepackage{amssymb}
\usepackage{array}
\usepackage[hidelinks]{hyperref}
\usepackage[open,openlevel=1]{bookmark}
\usepackage{multirow,tabularx}
\usepackage{listings}
\usepackage{color}
\usepackage{tikz}
\usepackage{float}
\usepackage[nottoc,numbib,notlot,notlof]{tocbibind}
\usetikzlibrary{positioning}
\usepackage{multicol}

\linespread{1}

\definecolor{maroon}{rgb}{0.5,0,0}
\definecolor{darkblue}{rgb}{0.0,0.0,0.6}
\definecolor{darkgreen}{rgb}{0,0.5,0}

\lstset{
	basicstyle=\ttfamily,
	columns=fullflexible,
	showstringspaces=false
}

\lstdefinelanguage{XML}
{
	morestring=[s]{"}{"},
	morecomment=[s]{?}{?},
	morecomment=[s]{!--}{--},
	commentstyle=\color{darkgreen},
	stringstyle=\color{darkblue},
	identifierstyle=\color{darkblue},
	keywordstyle=\color{maroon},
	morekeywords={dependency,groupId,artifactId,version,exclusions,exclusion,jsp,include,param}
}

\newcolumntype{C}[1]{>{\centering\arraybackslash}p{#1}}

\hypersetup{linktoc=all}

\begin{document}

%----------------------------------------------------------------------
% Title

\begin{titlepage}
\pagenumbering{gobble}

\begin{figure}[H]
    \begin{center}
        \includegraphics[width = 0.4\textwidth]{img/uob-text.jpg}
    \end{center}
\end{figure}
\begin{figure}[H]
    \begin{center}
        \includegraphics[width = 0.2\textwidth]{img/uob-seal.png}
    \end{center}
\end{figure}


\begin{center}
    \large
    \vspace{0.5cm}
    Final Year Project\\
    \vspace{0.5cm}
    \huge
    \textbf{Extracting Key Phrases and Relations from Scientific Publications}\\
    \large
    \vspace{1cm}
    Dissertation for B.Sc in Computer Science\\
    \vspace{.5cm}
    School of Computer Science, University of Birmingham\\
    \vspace{2cm}
    Author\\
    Thomas Clarke (1443652)\\
    \vspace{0.5cm}
    Supervisor\\
    Dr Mark Lee\\
    \vspace{2cm}
    \textbf{April 2018}\\
\end{center}
\end{titlepage}

%----------------------------------------------------------------------
% Contents and stuff
\pagenumbering{arabic}

\section*{Declaration}
The content of this dissertation has been produced by the author as part of the BSc Computer Science at the University of Birmingham. None of the material has previously been submitted for a degree at the University of Birmingham or any other university. The research conducted has been by the author unless indicated otherwise.

\pagebreak

\section*{Abstract}
This project presents solutions developed to solve the SemEval 2017 ScienceIE task - analysis of scientific publications to extract key information. This includes three subtasks: \textit{(A)} key phrase extraction, \textit{(B)} classification and \textit{(C)} relation extraction.

To achieve subtask A, the text of a paper is parsed to find it's semantic tree. Then, each word in succession is tested in a Support Vector Machine (SVM), based around a words' semantic attributes to determine if it should be a, or part of a, key phrase. Each phrase generated is also sanitised to reduce excess information. Clustering based off of the Word2Vec distances between words was also experimented with, but was not able to produce satisfactory results. Subtask B involved treating each key phrase as a Bag-Of-Words, and calculating the phrases' distance to each classification type using Word2Vec. Finally, subtask C experimented with using the Word2Vec representation of a phrase and the relative distances between phrases combined with an SVM to try too detect relations.

The best solutions found in this paper for subtasks A, B and C, under the ScienceIE script evaluation, saw F1 scores of 0.2, 0.11 and 0.02 in end-to-end tests and scores of 0.2, 0.55 and 0.1 when tested individually, respectively. 

To explore how this system could be used, a proof-of-concept website was created hosting the information. This used Spring Boot to create a Java based web project which supported not only an archive of processed papers, but also the means to search using query strings and automatic processing of submitted papers to the system (through using the most successful versions of systems described above). The search was the main feature developed, which aimed to use the key phrase information and the phrases' tokens' TF-IDF scores to help prioritise the results more relevant to the user if they use key words from the target papers. A reasonable solution was found for this, although further testing with a larger range of papers would confirm its usefulness. Two data visualisations, a donut chart and a set of word clouds, were also implented to display snapshots of the information extracted by the earlier systems. \\

\noindent All code and resources produced and used throughout the production of this project are available at:
\begin{center}
	\texttt{\href{https://git-teaching.cs.bham.ac.uk/mod-ug-proj-2017/tbc452}{https://git-teaching.cs.bham.ac.uk/mod-ug-proj-2017/tbc452}}
\end{center}
\noindent Appendix \ref{appendix:howtorun} gives instructions on how to run the program. Similar instructions are also available in the \texttt{README.md} file at the root of the Git repository listed above.

\section*{Keywords}
Natural Language Processing, Information Extraction, Classification, Relation Extraction, Support Vector Machine, Word2Vec, Java, Spring Boot

\section*{Acknowledgements}
I would like to give acknowledgement to those who helped me throughout the completion of this project.

Firstly, a thank you to Dr Mark Lee for being a supportive and informative supervisor, as well as an entertaining host during project meetings.

I also wish to thank my friends and family in supporting me during the year leading preceding this dissertation, ensuring I kept on track and in a good frame of mind while working.

\pagebreak

\tableofcontents
\listoffigures
\listoftables
\pagebreak


%----------------------------------------------------------------------
% Parts

\section{Introduction}


\subsection{Aims and Objectives}

\subsection{Report Outline}

\pagebreak
\chapter{Background and Literature Review}

\section{Definitions and Descriptions}
Throughout this literature review there are several key natural language processing and machine learning concepts discussed. Rather than defining them as we arrive at them, a list of useful definitions is constructed here.

\subsection*{F1 Score}
The F1 score is a metric used to evaluate predictions. A common concept to find whe evaluating binary decisions is a \textit{confusion matrix} from which ROC analysis can be completed \cite{Fawcett2006}. This is a system which records \textit{true positive} and \textit{true negative} where the gold standard and predicted data match, and \textit{false positive} and \textit{false negative} where gold and predicted data do not match. From this, various values can be calculated. 

Accuracy is one, but often doesn't show the full story as if, on a data set where 9 out of 10 items are 'false' and just 1 item is 'true', predicting all false will get an accuracy of 90\%, but no \textit{true positive} occurrences will appear - which is bad. 

Two better metrics can be calculated, which are precision and recall. They look at the rates of correct and incorrect predictions, and can be combined together (and often are in the NLP world) to produce an F1 score. This is what ScienceIE's scripts calculate, given ScienceIE's gold standard data and a researchers predictions, comparing instances where the researcher has correctly predicted key phrase boundaries, classification and relations against the gold standard data.

An F1 score of 1 is perfect, and an F1 score of 0 means nothing was classified correctly at all.

\subsection*{Tokenization}
Tokenizatoin is a simple concept where a document is broken down, from one long string into individual words or symbols. 

\subsection*{Bag-Of-Words Representation}
Bag-of-Words is another simple concept, where each token is considered independently of is semantic meaning. 

\subsection*{Stop Words}
Stop words are words that are commonly filtered out due to their lack of specific meaning and generally do not contribute to useful input, although exceptions can be made if there is little other information. For example, common search engines are likely to ignore the word "the" in most searches, but this process of removal needs to be conducted carefully as, for example, when searching for "the who", "the" is an important part of that query. 

The stop words concept is used in this report, and the list of stop words used is taken from the Stanford CoreNLP GitHub repository\footnote{\href{https://github.com/stanfordnlp/CoreNLP/blob/master/data/edu/stanford/nlp/patterns/surface/stopwords.txt}{https://github.com/stanfordnlp/CoreNLP/blob/master/data/edu/stanford/nlp/patterns/surface/stopwords.txt}} and shown in appendix \ref{appendix:stopwords}.

\subsection*{Term Frequency - Inverse Document Frequency (TF-IDF)}
TF-IDF is a metric for assessing how important a word is in a piece of text and has many applications in NLP, including (which will be discussed later) document query \cite{Ramos2003}. It shall also see use throughout this report.

To be calculated for a given token, the document that token came from is required and a set of documents for comparison is required. The set of documents used in this project shall be the ScienceIE training set.

The theory is, a word like "the" - which will likely appear many times in almost every document - should a very low TF-IDF score (close to 0). Unusual words however, such as "xylanases" which are likely very specific to the paper they are contained in will probably have a much higher TF-IDF value than that of "the".

\subsection*{Parse Trees and Part-Of-Speech (POS) Tagging}
A \textit{parse} of a sentence is a tree structure where the root node is a \textit{sentence}, each node below that is a \textit{POS tag} and the leaves of the tree are the words or symbols in the sentence. The POS tag tells us about the semantic meaning of that node and its children - or sub tree - where the POS tag could be \textit{verb phrase} or \textit{noun phrase} for example.

\subsection*{Support Vector Machine (SVM)}
A SVM is a supervised machine learning mechanism. The input to a SVM is a series of vectors generated from some original input data. Each of these vectors is a set of features - which are simply values calculated based on the original input data. For training, these data points are labelled, indicating their class. Once trained, a SVM can be used to \textit{predict} the label for a new data point.
% todo a refernce here and improve in general

The training involves attempting to find a well fitting hyperplane with a maximal margin, that separates the labelled data, after mapping that data into a higher dimensional space. This involves an algorithm for finding the distance between the mapped data points, for which a \textit{kernel} can be specified. Furthermore, the hyperplane that is fitted can be allowed to make errors. This is where it allows training data points to be within the hyperplane margins (so may be miss classified if tested against). This can be tuned to increase SVM performance at the cost of run time increasing as well.

\subsection*{Clustering}
Clustering is simply the idea of grouping items together that are similar. The result should be a set of sets of items, where within the group there is a high average similarity, while inter-group similarity is much lower. This is often used for classification and there are a variety of clustering algorithms that can be applied, with various algorithms performing better for various applications \cite{Rai2010}.

\section{ScienceIE Proceedings}
Evaluating the outcome of ScienceIE at SemEval indicates potential paths for future systems and documents very recent activity in the information and key phrase extraction area. Three papers were published from the event regarding this task.

Firstly, an overview of how successful the task was shall be conducted. The highest end-to-end F1 score achieved by any team was measured to be 0.43 for all three sub-systems combined, with each of subtasks A, B and C were 0.56, 0.44 and 0.28 respectively (for end-to-end tests) \cite{Augenstein2017}. 

For subtask A, it was evaluated that while many high scores were achieved with recurrent neural networks, the highest scoring system was a SVM using a well-engineered lexical feature set. SVMs and neural networks were also popular choices for subtask B. For subtask C, many methods were attempted and while a convolutional neural network was the most effective, various other methods (including SVM, multinomial naïve Bayes and Conditional Random Fields (CRFs)) all achieved very similar and reasonably accurate scores (the best had an F1 score of 0.64 when evaluated solely on subtask C).

Furthermore, a common preprocessing technique was to use the spaCy NLP pipeline to analyse the given texts to achieve knowledge about their semantics \cite{Honnibal2015}. Then, the key phrases were directly annotated on to the text model with a flavour of tagging scheme (typically Inside-Outside-Begin or Begin-Inside-Last-Outside-Unit) so that algorithms operating on the produced models have the annotation information to learn from. Algorithms would then predict these tags on the test data, which would then be post processed into the BRAT annotation format for evaluation.

The best end-to-end ScienceIE team applied sequence tagging models, which in tern used long short-term memory (LSTM) neural networks and CRFs to solve each subtask separately \cite{Ammar2017}. Their sequence tagging models also employed gazetteers built from scientific words extracted Wikipedia and Freebase, which appear to perform suitably in the context of scientific papers, as their high results (scoring first or second in all of the scenarios evaluated) indicate the vocabulary supported by these two resources covered much of the test data (their algorithms would likely have not scored so high if a large portion of the vocabulary was missing from the gazetteers created).

Another contribution also included the use of CRF based models \cite{Marsi2017}. This team actually created a range of CRFs, each with a different intention. They built a range of models, each targeting a specific goal (general key word extraction, task key word extraction, task classification and so on), and each with their own set of features chosen through cross validation. These features generally focussed on the format of the word, the position of the word, and (for material key phrases) comparison to known hyponyms, using WordNet\footnote{\href{https://wordnet.princeton.edu/}{https://wordnet.princeton.edu/}} as a data source. A problem encountered here was that they found some classes had few examples given, so as a solution they removed all sentences that didn't contain a given class (to only focus on areas with positive examples to help balance class distributions). Finally, as there were overlap in the CRF classifiers created, they could be run in parallel and a voting system used to choose a final prediction.

Both of the solution sets above also used sensible rules to help improve their score, such as intuitively marking all instances of a key phrase as a key phrase upon finding one instance (so if \textit{carbon} is extracted and labelled as a \textit{material}, then all other instances in the same document are labelled to match). The teams also exploited hyponym relationship’s bidirectional property (so if word 1 is a hyponym of word 2, word 2 is a hypernym of word 1), meaning their classifier could classify the relation in either direction and the final \textit{hyponym-of} relation could be returned.

The results of ScienceIE demonstrate there are several potential systems that could be implemented to answer this problem, with the best system potentially being a combination of algorithms to select and label key phrases, either directly working together or separately and then their results being passed through some voting system to choose a final prediction. The majority of the solutions presented at ScienceIE involve supervised learning, which requires the use of the training data, with little mention of unsupervised approaches. This suggests the current best solutions require learning from examples, which could cause a problem if the training data is of poor quality or (as one team worked around) there a lack of examples for a given class. Furthermore, as training data is being used, over-fitting needs to be considered, where solutions presented may work well for the test set but not in the real world, but unfortunately it is hard to evaluate the application of the produced algorithms on random scientific articles as the researched ended after testing with ScienceIE data in most cases.

\section{Further Background}
Now a review of research in a larger field is reviewed to explore other contributions to information extraction.

A CRF based key phrase extraction model was completed for use on Chinese documents which saw very high results. Before describing their findings, this paper targets a language other than English, so their use of semantics may have different affects if applied in the same way to English sentences. The CRF model designed contained 22 different features, where almost all of the features were based around the position of the token in the text. For their test data set, the largest F1 score measured was 0.51, which is very similar to the best scoring mechanism used on the ScienceIE data set. This proves CRF models can be effective when applied to multiple languages, but also infers that little progress has been made in the decade since this research, given (although different data sets were used) very similar results. 

As mentioned earlier, WordNet is also a potential source of information for building a classifier, although other knowledge bases exist. One study compared building a classifier using WordNet and Wikipedia for hyponym-only extraction \cite{Snow2013}. The concept of the system produced was that it could learn the lexical patterns of example hyponym pairs so that could be used to find hyponym relations. Using Wikipedia scored higher (F1 score of 0.36 on their dataset) than when using WordNet (F1 score of 0.27), showing it may be more suited to creating this type of classifier. While an improvement, it is not ultimately a huge increase and there is no evidence either Wikipedia is more useful for the specific area of scientific papers, as the dataset used in this study was not focussed. However, Wikipedia has more information that WordNet, and would likely cover more of the scientific concepts covered by any ScienceIE data. It is also updated on a more regular basis meaning new definitions are likely to be available much earlier than if using WordNet. With the study being completed in 2013, as Wikipedia matures it may, over time, increase or decrease the result of the experiment if repeated, depending upon the quality of the information contained in Wikipedia, which if tested would help evaluate its effectiveness.

A method not attempted at ScienceIE was unsupervised learning by clustering key phrases, a method which has potentially very accurate results that also could not only be robust again new unseen data but even different languages. A study used the idea that words can be clustered together through relatedness, with the goal of producing a set of clusters representing topics in a document \cite{Liu2009}. To extract the key phrases, using the simplest approach, the centre of a cluster is the key phrase. Various clustering methods were attempted by this study. They ran tests on relatively short articles and while at maximum they only achieved an F1 of ~0.45, there was several improvements suggested which apply to the ScienceIE task concerning scientific papers. Two suggested covered how the clusters were formed, including clustering on noun groups (which most key phrases at ScienceIE are) and creating a heuristic for co-occurrence based and Wikipedia based relatedness. Another suggestion was to improve the removal of unimportant words. The study only used stop words, but introducing a TF-IDF filter could also be possible.

\section{Word2Vec}
Through recommendations, this project also considers the use of Word2Vec. Word2Vec is an application of neural networks, that processes text and creates a vector space containing each word in the text. This vector space can be used to find the similarity and differences between words \cite{Mikolov2013}. This technology is relatively new (initially designed in 2013) and its uses are still being discovered, with a similar project called GloVe being worked on the following year \cite{Pennington2014}.

Given a large set of input texts, Word2Vec aims to learn the meaning of words. It processes all examples of where every word in the document library has been used contextually, and builds a vector representation of each word. These vector representations can be compared to try to extra the similarity, or relative similarity between words. For example, the relative distance from a \textit{knee} to a \textit{leg} may be similar to that of \textit{elbow} to \textit{arm}. It has been proven to be effective, although very difficult to understand \cite{Goldberg2014}. 

In terms of information extraction, there has been a recent study into its use to find extra hyponym information from text \cite{Nayak2015} using GloVe. This study designed a function based on the GloVe vector space to create a system to predict hyponym pairs. Reasonable results were achieved, scoring up to 35\% precision in some cases. However, a problem discussed is that words can have multiple meanings, which can effectively confuse the model, which decreases its quality for determining similarity and therefore reducing the effectiveness of a system built. 

\section{The Supplied Data Set}
The ScienceIE data set consists of 50 development, 350 training and 100 test documents. 

Some analysis conducted at ScienceIE \cite{Augenstein2017} showed some characteristics of the sample key phrases included:
\begin{itemize}
	\item Only 22\% of key phrases had 5 or more tokens,
	\item 93\% of key phrases were noun phrases,
	\item Only 31\% of key phrases seen in the training set were also in the test set.
\end{itemize}

This means that key phrase extraction appears quite difficult, as an algorithm needs to search for short phrases, processing phrases that it likely hasn't seen instances of before. Most of the key phrases being noun phrases, however, is valuable information as it helps to identify a simple heuristic that can be used when processing.

Other useful and interesting characteristics about the training set, found during this study, can be seen in table \ref{table:traininganalysis}. 

\begin{table}
	\centering
	\begin{tabular}{ c | c c c c }
		& \textbf{Minimum} & \textbf{Average} & \textbf{Maximum} & \textbf{Standard Deviation} \\
		\hline
		Nnumber of KPs & 4 & 19 & 46 & 8 \\
		Minimum tokens per KP & 1 & 1 & 3 & 0.4 \\
		Average tokens per KP & 1 & 3 & 8 & 1 \\
		Maximum tokens per KP & 2 & 9 & 25 & 4 \\
		Number of relations & 0 & 2 & 13 & 2 \\
		Total tokens in document & 60 & 159 & 264 & 46
	\end{tabular}
	\caption[ScienceIE Training Set Analysis]{Key phrase (KP), token and relation analysis for the ScienceIE training set.}
	\label{table:traininganalysis}
\end{table}

Papers in the ScienceIE data set have many key phrases associated with them. With an average of 19 key phrases per paper, an average of 3 tokens per key phrase (meaning on average 57 key tokens per paper) and the average document containing only 159 tokens in total, around a third of all tokens are part of key phrases. This is partly due to the documents supplied by ScienceIE being very short (all are just one paragraph) and are \textit{extracts} of papers rather than full publications. It is not a problem that the documents for processing are short - in fact that may help as the longer the document, the harder it is to choose key phrases \cite{Hasan2014} - however, it may mean any algorithm created here may not scale well to full scientific papers. It seems the ScienceIE task is looking for localised key phrases, choosing several from one paragraph; while the author of a paper may choose to select just five or ten key phrases from the whole paper. While this project will focus on the ScienceIE task with the given test data, a brief look longer or full papers shall be considered.

\pagebreak
\chapter{Project Architecture}
With any large software project, it is sensible to choose a platform with all the necessary tools available so the developer can achieve their goals. The following describes the environment and technologies used generically through out the entire project.

\section{Language}
Due to the past experience of the author, Java was an obvious choice. Given extensive time working in the language during university and in industry, a thorough understanding of the programming language was already achieved, which allowed for planning of a sensible software architecture to optimise code quality and (implicitly) the potential of increased success of the systems created. 

Furthermore, Java is a very popular and accessible language world wide - backed up by the active StackOverflow community (casual and professional alike) with Java being one of the most popular technologies for at least the last five years, evidenced through their user surveys 2018\footnote{\href{https://insights.stackoverflow.com/survey/2018}{https://insights.stackoverflow.com/survey/2018}}, 2017\footnote{\href{https://insights.stackoverflow.com/survey/2017}{https://insights.stackoverflow.com/survey/2017}} and 2016\footnote{\href{https://insights.stackoverflow.com/survey/2016}{https://insights.stackoverflow.com/survey/2016}}. Due to this, Java has extensive support for many common problems people encounter, with issues being discussed and solutions proved across various forums. 

Not only is Java's popularity good for increasing support availability, many libraries and utilities are available to help developers with tasks. Along side other technologies used for more specific tasks throughout completion of the NLP system and the POC system (which shall be discussed when used), common technologies used during the development of the entire project are described below. Throughout development of the project, very little issue was caused by lack of Java support for common processes or lack of Java capability when attempting to program some process (which was a critical part of evaluating which language should be used). 

Finally, Java serialisation was used throughout (and will be noted when is). Serialisation allows the system to save a Java object to disk (any file name can be chosen, but classically its postfix is \texttt{.ser}), and later be reloaded. 

As a brief aside, Python is another extremely popular language used for NLP and likely could have been used for at least the first half of this project producing similar results.

\subsection*{log4j}
log4j 2\footnote{\href{https://logging.apache.org/log4j/2.x/}{https://logging.apache.org/log4j/2.x/}} is a popular and robust library developed under the Apache Software Foundation to do logging in Java. It's useful features include:
\begin{itemize}
	\item Automatic output of logs to both terminal and file: As well as immediate visual feedback, log files can be used for later processing and evidence gathering.
	\item Timing of events: Timing is very useful as during long runs of a system (for example, some sections of the NLP task could take hours to complete) the logs can be analysed to see how long systems take to process data, which can be considered when going forward; for instance in terms of formulating efficiently timed tests plans.
	\item Labelling of logs into levels such as \textit{debug}, \textit{info}, \textit{error} and \textit{fatal} messages: This can be used when analysing the logs to catch where things went wrong (filtering for error messages) and then to try to debug the system by finding information logged prior to that (with debug). During development an excellent use of this feature is to output all levels aside from 'debug' to terminal, so monitoring progress isn't overloading the executor with information, but if something does go awry the steps leading up to the bad event can be analysed in the log saved to disk.
	\item Specified layout of logs: the developer of a project can detail what information (and the precision of the information) is included in a log statement (for example, time of the log, source of the log). This, along with all other configuration for using log4j 2, is completed in \texttt{log4j2.xml} in a Java projects \texttt{resource} folder.
\end{itemize}

While direct output of this will not be present in the rest of this report, it is worth noting this was an extremely useful tool for developing all of the systems to follow. 

\subsection*{Maven}
Apache Maven\footnote{\href{https://maven.apache.org/}{https://maven.apache.org/}} is another important tool. Like log4j, it is developed by the Apache foundation. 

Maven is a tool to help with project management and has many uses. It is based around a \textit{project object model} (POM) configured in a \texttt{pom.xml} file at the root of a Java project, which itself has a structure defined by Maven. The key uses utilised in this project are:
\begin{itemize}
	\item Project compilation: Maven can be used to build a project and automatically run specified or all tests, with more detailed and well formatted output than compiling Java code by hand. Therefore, compilation and testing can more easily be scripted and output more clearly analysed. It also handles importing libraries used in a Java project when compiling (which can be very troublesome when completed by hand), which is discussed below.
	\item Library import: The \texttt{pom.xml} can specify dependencies of the Java project. While custom, third party repositories exist, Maven has a central repository\footnote{\href{http://repo.maven.apache.org/maven2/}{http://repo.maven.apache.org/maven2/}} with many libraries available. This includes log4j described above, and all other libraries used in this project. Dependencies are downloaded to the systems local Maven repository at compile time.
	\item Library export: As discussed in the introduction, the NLP system shall be used in a POC system. Rather than combining these two systems into one large package, or doing a confusing copy of the required resources, Maven can be used to export the compiled NLP system to the local Maven repository. Then, the POC system can simply list the NLP system as a dependency, and Maven shall include it as a library when building the executable program.
\end{itemize}

Maven is used as the management backbone throughout the development of software discussed in this report. When libraries are used in a project, a link to their dependency configuration for Maven's \texttt{pom.xml} shall be included. As a good example, log4j\footnote{\href{https://logging.apache.org/log4j/2.x/maven-artifacts.html}{https://logging.apache.org/log4j/2.x/maven-artifacts.html}} has an extensive page providing a detailed description of how to import the library.

\subsection*{JUnit}
JUnit is a popular Java framework for testing. It is simple to use, catching unexpected (or expected) exceptions and ensuring values are correct with \texttt{assert} statements. 

Maven also integrates with it, so that (by default) when you build a Java project with Maven, all of the methods marked with \texttt{@Test} annotation in the test source directory are executed to ensure the program is working as expected (as far as the tests ensure that). It will then provide a report and trace of any issues once complete. Maven will also automatically exclude the test files from the final packaged product to reduce waste space for deployments of projects. 

While working through this project many JUnit tests were constructed (all of which are still available in the Git repository for this project). Somewhat unconventionally, there is a divide between tests: while some are based around ensuring functionality works as expected, many are actually building the NLP systems, training them (if required), testing them and comparing the predictions made to the gold standard data.

The tests can also be ignored\footnote{\href{http://maven.apache.org/surefire/maven-surefire-plugin/examples/skipping-tests.html}{http://maven.apache.org/surefire/maven-surefire-plugin/examples/skipping-tests.html}} which is very useful, as many of the tests written are base around evaluating the algorithms created rather than testing functionality; so not only does not every algorithm need to be retested at every compilation time, but if they were it would take many hours (and probably more memory than the standard computer has) to build and test the application. 

\subsection*{Word2Vec}
The interesting Word2Vec technology is utilised in this project in various places. The original Word2Vec library implementation was in Python. However, the Deep Learning For Java (DL4J) team have included, as part of their machine learning and deep neural network library, Word2Vec funcitonality\footnote{\href{https://deeplearning4j.org/word2vec.html}{https://deeplearning4j.org/word2vec.html}}. This supports training a Word2Vec model, using the model, and saving and loading models. 

The models used in this project are the Google News model\footnote{\href{https://drive.google.com/file/d/0B7XkCwpI5KDYNlNUTTlSS21pQmM/edit?usp=sharing}{https://drive.google.com/file/d/0B7XkCwpI5KDYNlNUTTlSS21pQmM/edit?usp=sharing}} and the Freebase model\footnote{\href{https://docs.google.com/file/d/0B7XkCwpI5KDYaDBDQm1tZGNDRHc/edit?usp=sharing}{https://docs.google.com/file/d/0B7XkCwpI5KDYaDBDQm1tZGNDRHc/edit?usp=sharing}}. While neither of these are made up of scientific articles, they both have a large vocabulary size (3 million and 1.4 million tokens respectively), and both based off of a 100 GB large samples, which should allow them to perform relatively well. This is the reason Word2Vec is being used over GloVe, as these models have much larger vocabulary sizes that the pre-trained GloVe models found, so in theory have a better change of covering the vocabulary of the ScienceIE data set\footnote{This GitHub page contains a selection of popular models for both Word2Vec and GloVe: \href{https://github.com/3Top/word2vec-api}{https://github.com/3Top/word2vec-api}}.

Attempts were made to use a Wikipedia based model (Wiki2Vec\footnote{\href{https://github.com/idio/wiki2vec}{https://github.com/idio/wiki2vec}}) but unfortunately no successful attempt was made to use it in this project (there were various problems converting the model to a Java readable format and loading it). While potentially of lower quality semantics (as Wikipedia isn't officially maintained) it may have had more of the vocabulary the ScienceIE data supports as Wikipedia covers many topics including those of scientific nature so could have increased coverage of the model when finding similarities between various scientific tokens. 

\section{Platform}
While Java is cross platform (another excellent reason for using it), some of the underlying system libraries that Wor2Vec relies on to function are included by default in many Linux distributions. It can be made to work on Microsoft Windows operating systems, but it requires a large amount of complex configuration and generally not worth the pay off. Therefore, this system was built on the Ubuntu 16.04 distribution of Linux, as this involved the least amount of configuration to get working.

Furthermore, some of the algorithms created as part of this paper are able to fill up available memory on a computer very quickly. The memory available as part of this project was 16 GB. Linux swap space was configured (an \textit{overflow} area for memory usage) but generally one would not like to use this, as it is slow to read from and will add some wear to the solid state drive in the host system available (due to many fast reads and writes) which isn't good. This is another reason for using Linux as the platform to build these systems on, as the memory overhead from the operating system is much smaller when compared with Microsoft Windows 10 (in the order of gigabytes of memory saved). Furthermore, Linux can also be run headless (without a GUI) to further reduce the operating systems memory usage, which on Ubuntu 16.04 saves approximately an extra gigabyte of memory. With support for Secure Shell (SSH) to remotely connect to the system, to run tests and read results, Linux is an excellent choice of platform for optimising memory usage while running these algorithms.

\pagebreak
\chapter{The ScienceIE Task: Design and Implementation}

To complete the ScienceIE task, the plan was made to have one Java project containing three sub systems, where each of which could be called independently. As such, this section shall step through each subtask's design and implementation in order, beginning with a description of the preprocessing that was implemented, as it is generic to all subtasks.

After this section has finished describing the design and implenetation of the varous algorithms used in this project, the following section shall describe the results associated with them.

\section{Data Preprocessing}
To support processing in later systems, all data (development, training and test) had to be preprocessed. The idea of this piece of computation is to prepare the data for analysis, and to also reduce computation time (doing this process once for the entire system rather than once for each sub system). To further reduce experiment run time, Java serialisation was also used to save all the following preprocessing information for later retrieval.

In Java, for each paper file from ScienceIE, a \texttt{Paper} object was constructed. This held many important pieces of information about the paper in question, including location on disk, text extracted from its source file, and all preprocessing information. \texttt{Paper} itself is a \textit{plain old Java object}, only holding information and is an abstract class, with \texttt{TextPaper} and \texttt{PDFPaper} classes extending from it which could be instantiated. These extended classes inherited the data storage features and utilities from \texttt{Paper}, but their constructors are customised to extract information from their given type of file:

\begin{itemize}
	\item \texttt{TextPaper} is for \texttt{.txt} files and simply extracts the text from the document. It sees the title of the text document as the title of the paper.
	\item \texttt{PDFPaper} is for \texttt{.pdf} files. This uses Apache PDFBox\footnote{https://pdfbox.apache.org/} (imported through Maven\footnote{https://pdfbox.apache.org/2.0/dependencies.html}) to extract the text from a PDF. The title, once again, is the title of the document. As alluded to earlier, with the ScienceIE test set not only being just text files but also being short documents, longer PDF papers was not usually used, so little development to properly sanitise PDFs happened, meaning all titles and references were also captured in this text extraction. If more PDF files were to be processed this would have been looked at, however, due to its lack of use the time needed to fix this was deemed not worth it.
	\item It was initially planned that there would be a \textit{HTML} and \textit{WebPDF} classes although, for similar reasons to why the \texttt{PDFPaper} text extraction was not developed further, these two classes were never implemented. The main reason for wanting them was to later support the POC system, as this would allow that system to dynamically grab papers from the web and add them to itself. Importing of papers to the POC system shall be discussed later at a more relevant time.
\end{itemize}

The bulk of the preprocessing came in the form of using a parser to calculate the parse tree of a text. As discussed, many teams at ScienceIE used spaCy. As the plan for this project was to complete it in Java (creating a single, self contained system) the Stanford CoreNLP package was used \cite{Manning2014} (imported through Maven\footnote{https://stanfordnlp.github.io/CoreNLP/download.html}). While offering a range of useful NLP features, the main ones utilised by the project were tokenization and finding the parse tree of the text (which naturally included POS tagging). An \texttt{Annotator} class was constructed which accepted a \texttt{Paper} input and annotated the text contained using the CoreNLP library.

Further processing on this information was also completed, where (at the time of saving the CoreNLP parse information) a token \textit{map} was created. This \textit{map}'s key set was all tokens present in the document, with the associated value being the number of times the token was in the document. This was to help when calculating TF-IDF scores later in processing.

The final part of preprocessing was to load existing annotation information. Of course this was only possible for ScienceIE data, which were all supplied with the relevant \texttt{.ann} files in BRAT format. These records were loading into a list of \texttt{Extraction} abstract entities, where each entry to the list could be either of a \textit{KeyPhrase} or \textit{Relationship} extending type, which each held all the information supplied in the annotation files (including classifications, the types of relations and more).

\section{Subtask A - Key Phrase Extraction}
Subtask A at ScienceIE was considered the hardest, reinforced by both the maximum and average scores for each independent subtask. This paper dedicated most of its NLP effort to this task out of the three subtasks as this is currently the hardest part of information extraction (out of the given subtasks) under current research. 

Two attempts at this subtask were made. Initially, a \textit{safer} design involved a SVM which considers some of the key features about key phrases suggested in the literature around this topic. Then, an even more experimental trail shall be described which involves clustering based around Word2Vec similarities between words in a document. 

\subsection{Method 1: Support Vector Machine}
Inspired by the highest success at ScienceIE, a SVM approach was adopted to attempt to provide a solution to subtask A. Initially, a small set of support vectors were selected and tested, with more being added as research continued.

\subsubsection*{Processing Data}
Two approaches were considered when designing the input and output data. One was based around passing each token in individually and in order, while the other was based around using the parse information obtained by using CoreNLP to pass sections of a sentence. 

Working with each individual token was selected for several reasons. Firstly, it was very easy to simply iterate through every token in a document in turn. Furthermore, the CoreNLP data is still available (evidences as that is what returns the tokens of the document) and can be passed to the SVM to be used when calculating support vectors. While using sections of a sentence should help keep any key phrase extracted more semantically correct (i.e. it should avoid missing the end of a noun phrase by accident which a check could be added for anyway), it poses a large issue: Any section selected as a key phrase would likely be \textit{locked down} as such to the specific tokens inside that section, meaning there may be no way to get rid of excess information or added extra if the gold standard key phrase requires something slightly different to the key phrase chosen by the SVM. In terms of extra information needed, a system could be implemented to join adjacent key phrases but that would like see extra information over what is needed being included. If, to try and solve this issue, some system which could extend or retract by a token or two was implemented, it is getting closer to the original option anyway where the system is processing the entire document as individual phrases. Therefore, a system based around processing each token individually was decided upon.

This resulted in a total of 65447 different training points (the total number of individual tokens in all of the training data).

\subsubsection*{Defining Support Vectors}
It is clear that current trends view the position of key phrases are very important in the document and should definitely be considered when trying to learn how to predict them. A tokens proximity to other tokens semantically and as part of the document as whole seem to significantly help us identify where key phrases lie. Furthermore, some attributes about individual phrases also seem to play a large part. For example, the length of the word is a valid feature to evaluate, as the average length of a key phrase token (7 characters) is slightly different to the average of all key phrases (8 characters). 

\begin{table}
	\centering
	\begin{tabular}{ C{7cm} | c }
		\textbf{Support Vector Description} & \textbf{Value Range} \\
		\hline
		The length of the token divided by the maximum token length in the training set. & \texttt{svLen} $\in$ $\mathbb{R}$,  0 $\leq$ \texttt{svLen} $\leq$ 1 \\
		\hline
     	Whether the token is a noun (using Part-Of-Speech tagging). & \texttt{svPos} $\in$ \{0, 1\} \\
     	\hline
     	The TF-IDF score of the token. & \texttt{svTfIdf} $\in$ $\mathbb{R}$,  0 $\leq$ \texttt{svTfIdf} $\leq$ 1 \\
     	\hline
     	The token index divided by the number of tokens. & \texttt{svDepth} $\in$ $\mathbb{R}$,  0 $\leq$ \texttt{svDepth} $\leq$ 1 \\
     	\hline
     	The token index in the current sentence divided by the number of tokens in the sentence. & \texttt{svDepthSentence} $\in$ $\mathbb{R}$,  0 $\leq$ \texttt{svDepthSentence} $\leq$ 1 \\
     	\hline
		Whether the token is in the first sentence of the paper. & \texttt{svFS} $\in$ \{0, 1\} \\
		\hline
     	Whether the token is in the last sentence of the paper. & \texttt{svLS} $\in$ \{0, 1\} \\
     	\hline
     	Whether the previous token was part of a key phrase. & \texttt{svLWKP} $\in$ \{0, 1\} \\
	\end{tabular}
	\caption[Initial Key Phrase Support Vectors]{Initial key phrase support vectors used. A set of these support vectors is generated for each token. When defining the value range, the variable is named as it is in the Java code.}
	\label{table:kpinitsvs}
\end{table}

Thankfully, the idea behind using an SVM is to find what separates key phrases from just normal phrases. Therefore, I was able to create an initial range of support vectors, as defined in table \ref{table:kpinitsvs}. Here it is evident most support vectors are based around trying to gather information as to the whereabouts of the token. It also, importantly, considers the sequence of key tokens.

\subsubsection*{Training}
To train the SVM, a \textit{problem} must be created. The \textit{problem} contains an array of data points, each of which holds a set of support vectors. Each of these data points must be labelled. The label is what we are trying to predict on the test data, so here the label is where or not the token is a key phrase (\texttt{0} for \textit{normal}, or \texttt{1} for key phrase). 

\subsubsection*{Model Selection}
As the nature of the data is unknown, an educated guess can be made as to which kernel to use. A common kernel to begin working with is the \textit{Radial Basis Function} (RBF) kernel \cite{Chih-WeiHsuChih-ChungChang2008}. This is because it can handle non-linear data, which it is assumed the training data here is to be. The RGF kernel function to find the similarity between two data points is listed below:

\begin{equation*}
K\textsubscript{RBF}(x_i, x_j) = exp(-\gamma||x_i - x_j||^2)
\end{equation*}

\noindent There are two parameters which can be configured and tuned to optimise performance of the SVM:
\begin{itemize}
	\item The cost \textit{C} parameter. This influences the misclassification allowance, where a small value lets the SVM select a large hyper-plane for separating data but allows for more misclassification, and a large value will allow the SVM to attempt to find a smaller hyper-plane that has less misclassification. Several values will be explored, of the set \{5, 50, 100, 200\}.
	\item The RBF kernel has a single parameter $\gamma$. From the same source that recommended the RBF kernel, as a initial value the SVM shall be configured to 0.5, as this is 1 divided by the number of features (we have 2 labels). However, other values shall be explored to attempt to find the best accuracy, and these values shall be \{0.25, 0.5, 1\}.
\end{itemize}

\subsubsection*{Development}
Having decided on how to use the concept of an SVM, there was a need for a concrete implementation. The idea of implementing an SVM was considered, however, with a responsibly high implementation complexity and a high risk of getting something subtly wrong (therefore being hard to detect and fix) a pre-existing solution was searched for. Furthermore, with the author having never handled an SVM before, a pre-existing solution with additional usage information was desired.

A popular SVM package was found in libsvm (imported through Maven\footnote{https://mvnrepository.com/artifact/com.datumbox/libsvm/3.22}). Originally written in C and ported over to Java (as well as many other languages), libsvm was designed to be flexible, supporting various kernels and suitable for beginners through to advanced users. It has support for the core use of SVMs - training and predicting, but also has features to aid in parameter selection such as a cross-validation function (which will be covered in more detail shortly), a visualiser for the training data and a data scaling tool.

% TODO add more about development

\subsubsection*{Cross Validation}
Cross validation is an important part when trying to optimise performance of an SVM. It allows for tuning key parameters by running repeated tests. Rather than using the testing data, which could introduce bias, the training data is split up into \textit{n} folds (or groups of data from within the training set). \textit{n = 5} folds were used in this instance. In turn, the SVM is trained with 4 of the 5 folds and then evaluated against the remaining fold. This is repeated for all combinations of folds and then the accuracy of the SVM can be calculated. A higher accuracy should mean better performance, although there is the problem of over fitting to consider. If the model is built to run perfectly on the training data, real world performance may actually suffer. This is why we cannot stop testing the SVM after just cross validation, as evaluating against the unseen test set will tell us how well it really performs. 

The values discussed for \textit{C} and $\gamma$ were used in cross validation and their outputs compared. 
% TODO add the kp svm cross validationgraphs
As evident in the graph showing the results of cross validation, very little change is accuracy is observed. There is a upward trend as \textit{C} and $\gamma$ increase, but this appears to plateu as higher values are evaluated. No smaller values need testing as (given the trend continues) the accuracy will significantly deminish. In terms of inrcreasing the values, there is potential for extremely small gains; however, the return from increasing these values will be almost worthless and the training time required as the \textit{C} value increase significantly rises, as the SVM is working harder to find a better fitting hyper plane. Therefore, \text{C = 100} and \textit{$\gamma$ = 0.5} shall be used when completing full experinments. While, a \textit{C} value of 200 is 0.05\% more accurate with the same $\gamma$ value, the training time is roughly doubled (from some hours to many hours) and the reward is not deemed worth it.

\subsection{Method 2: Clustering}
\subsubsection*{Concept}
Clustering has been shown to be effective in key phrase and other information extraction. A seemingly effective method was using \textit{term relatedness} to group terms, and then find an exemplar term at the centre of the clusters which can be used as a key phrase \cite{Liu2009}. 

Inspired by this, this paper proposes a similar process, where the similarity of terms is based around their Word2Vec similarities. Conceptually, it is possible that gathering similar terms will have the effect of creating clusters of important concepts, from which key phrases can be extracted. With the most similar words at the center, these could be considered key words, and the phrase around these words could be extracted from the document. This should also exclude unimportant or stop words, as these may have large distances to the key concepts. 

An important note here is that the document is treated as a \textit{bag-of-words}, where we ignore the semantic meaning of the words. Word2Vec does this anyway, as when using the library the word is simply passed as plain text with no extra information, and Word2Vec uses its own ideas about the words semantic meanings to evaluate it. Using a bag-of-words approach means words from any part of the document could be clustered together, which is ok as if both are close to the centre of a cluster, it may mean both of them are key phrase worthy and should be selected (a full phrase from their origin extracted to be a key phrase).

The clusteing algorihm selected was hierarchial clustering \cite{Rai2010}. Bottom up (agglomerative) hierarchial clustering works as follows:
\begin{enumerate}
	\item Each element begins in their own cluster (so \textit{n} elements means initially there are \textit{n} clusters).
	\item Given some distance metric, the distances between all clusters are calculated.
	\item The closest two clusters are combined.
	\item This process is repeated until a single cluster is left.
\end{enumerate}
\noindent This algorithm was chosen for several reasons:
\begin{itemize}
	\item Firstly, it is a reltively straight foward clustering algorithm to implement, so should allow results to be found quickly.
	\item While the term similarities are based off of Word2Vec similarities, the distances between clusters could be evaluated in a number of ways. These include \textit{single} (the shortest distance between terms in each cluster), \textit{average} (the average distance between each term in one cluster to each term in another) and \textit{complete} (the largest distance between terms n each cluster). These are called the \textit{linkage criteria}.
	\item If successful, the benefits of using clustering could be two fold. Not only might this produce effective key phrases, but it may even aid in classification. Once a number of clusters have formed key phrases, the hierarchial clustering could continue potentially all the way to producing just a few clusters where each could be classified into one of our 3 target classes. This will only be successful if the key phrase extraction is successful. If doing this clustering was purely for classifiaction, k-means clustering may be more appropriate with a \textit{k = 3} where each cluster would be a different classifiaction.
\end{itemize}

When working with hierarchial clustering, an important aspect to consider is how far to iterate through the algorithm. In theory, the algorithm should run until there is just one cluster left with every element in it - but this is not useful for anything. The more useful cluster states will be part of the way along the iterative cycle. To find this \textit{sweet spot} will require manual tuning, via inspecting the progression of the clusters to try to identify a good range where key phrase information can be extracted. In theory finding the sweet spot could be automated and learnt (by trying to extract key phrases at all levels and evaluating against the gold standard phrases to see how well the algorithm does) but to fully develop this kind of system would require a lot of time, which itself is very expensive, and if a failure it would be a big waste of time.

A large difference between this and SVM usage is that this method is unsupervised learning and does not require training data, while using an SVM is supervised learning. This means that, given this method works for this testing scenario, its application may scale better in the \textit{real world} as is it not tied to the quality of the training data. Furthermore, given a suitable WOrd2Vec model, differences in key phrase output may be seen. This means that this algorithm may suffer here given the Word2Vec models currently available are not based on scientific publications, but running this algorithm on test data with similar context to the Word2Vec model may improve things, which means it may even cope with different languages.

\subsubsection*{Development}
The process of turning this theory into a practical implementatin is straight forward. To allow for future expansion if clustering was to be used again for anything, a generic abstract \texttt{Cluster} class was created, which was formed of a list (where type is specified at creation time) of items and declaration of functions to find the distance between a given cluster object and another cluster, and to create a new cluster by combining a given cluster object with another.

A \texttt{Linkage} enumeration was created which could be used when testing to specify the method for finding the distance between two clusters.

Actually commencing the clustering begins by splitting the document into all of its tokens, and removing duplicates. Then, all stop words and unimportant words are removed. Unimportant words are calculated based on thresholding TF-IDF scores. This threshold was manually set, by generating a list of all tokens and their TF-IDF scores and evaluating where the cut off would be to remove all words not in any key phrase. For example, "WORD" has a low TF-IDF score of SCORE, but is in the key phrase "WORDS".

%TODO see if I have TF-IDF evidence for the above satement and find example

Then, the process of hierarchial clustering happens as described above. This happens for all test data, with all three linkage methods, with results being saved to disk. The output is then evalauted on a sample of the processed documents (as manually reviewing all 100 test documents cluster patterns accross 3 different linkage criteria is too much for a single reviewer to achieve) to try to evaluate where key phrases can be extracted from.

\section{Subtask B - Key Phrase Classification: Word2Vec}
With a large amount of influence coming from Word2Vec, the decision was made to focus on exploting this technology to try to classify key phrases.

\section{Subtask C - Relation Extraction: Support Vector Machine}
A section all about what I did for part 3
\subsection{Many Support Vectors}
Discuss the SVM I tried to do this with (including Word2Vec)
\subsection{Few Support Vectors}


\pagebreak
\chapter{The ScienceIE Task: Evaluation}

How each section went, including test results and maybe some info on other experiments.

\section{Subtask A - Key Phrase Extraction}

\subsection{Method 1: Support Vector Machine}

\subsection{Method 2: Clustering}

\subsection{Conclusion}

\section{Subtask B - Key Phrase Classification}

\subsection{Other Experinmentation}
As a small experinment, given the SVM for subtask A had undergone quite some development, 

\subsection{Conclusion}

\section{Subtask C - Relation Extraction}

\subsection{Conclusion}


\pagebreak
\chapter{Creating a Proof of Concept Use for ScienceIE data}
\section{Concept and Specification}
Having completed systems to handle the information extraction, the next major part of the project was to explore how this information can be utilised for researchers. 

The most obvious domain for using summary information about a set of resources is search. Being able to efficiently search through a large set of documents to quickly find more useful information is a very useful thing. The focus of this section is to explore using the key phrase information in an efficient and convenient way to aid in user query across the ScienceIE data set. 

The ScienceIE data set is specifically being used because that is what the algorithms in this project are designed for. The problem of handling longer, full publications has been discussed, so to not complicate this investigation further, the standard short documents of ScienceIE shall form the database of documents.

The following requirements are presented for the POC system to conform to are:
\begin{enumerate}
	\item As a basis to make the system usable, the ability to list papers in the database, and the ability to present all the information about a paper on one screen. The ability to download the extractions in the BRAT format should also be made available. 
	\item A search system for the key phrase information gained. This should allow the input of a user query (plain natural language text), with an option to give some indication of the classification of the information they are interested in (i.e. \textit{task}, \textit{process} or \textit{material}).
	\item The ability to add papers dynamically to the database throughout run time. This means not only should the system support the creation of new paper records, but also the automatic processing of that paper through all 3 subtasks of ScienceIE, with the results being available to the search system.
	\item Some way of conveying summary information about the extractions contained in the system. While not essential for search, having a convenient way for the developer and users to take a snapshot of what is contained in the system is useful for evaluation purposes, including giving an indication of how many papers are included and their processing status.
\end{enumerate}

\section{Background and Technology Review}
\subsubsection*{Existing Search}
Two popular, publicly available search engines have been examined to extra some well designed features from. These are Google Scholar\footnote{\href{https://scholar.google.com/}{https://scholar.google.com/}} and ScienceDirect\footnote{\href{https://www.sciencedirect.com/}{https://www.sciencedirect.com/}}, and appendix \ref{appendix:searcheg} shows screen shots of both web sites given the same query. Common features are:
\begin{itemize}
	\item Limited results per page (10 on Google Scholar, 25 on ScienceDirect),
	\item The total number of documents selected and the time it took to do this is displayed to the user,
	\item Filtering options, including by date,
	\item The title of a result is followed by a snippet of the document, with relevant words highlighted in some way,
	\item The title is a link to an individual results page which presents information about the document, but there are also links to directly get the document downloaded.
\end{itemize}

\noindent Some custom features on each website are:
\begin{itemize}
	\item ScienceDirect supports searches with multiple parameters, including query fields specifically for author, title, pages, etc...
	\item ScienceDirect also supports custom numbers of search results per page; 25 is the default, but the user can receive 50 or 100 results per page if they wish,
	\item Google Scholar present \textit{related searches} to allow the user to navigate sideways towards their target paper, rather than straight down to it,
	\item ScienceDirect features a \textit{recommended} section at the top where it showcases several publications it believes to be the most relevant.
\end{itemize}

Given the data set available in this project, most of these ideas presented by Google Scholar and ScienceDirect can be implemented. However, as ScienceIE's data is purely text snippets of documents, author, real title, publish information and more are unavailable, meaning any system created here cannot support filtering by these features, or display that information. An original hope of the developer was to include a more advanced search, potentially including a PageRank type system \cite{Page1998} where the links would be based on referencing, but unfortunately without even the reference information for each paper available, this is impossible to achieve whit the ScienceIE dataset.

The search on these two websites is likely be based on latent semantic analysis (\cite{Landauer1998} and \cite{AswaniKumar2012}) which looks at the similarities between search terms and attempts to take into account context. This method of representing words in a vector space prepared for querying is a popular method of document retrieval. This method of querying could potentially be modified to involved more weighting of tokens if the are included in key phrases, or be used to inspire a new algorithm that uses key phrase information to help with weighting word importance.

\subsubsection*{Available Technologies}
The goal with this project is to prove a use of the data produced as part of ScienceIE through some application. To make this application available for use, a web based service will be created that allows any user to join without the need for complex setup. The benefits of this include not having to distribute installation media and update media, along with centralised management of any database of information and being able to immediately make available new features.

In terms of storing information, popular database technologies were surveyed. The stack overflow survey 2017 listed MySQL\footnote{\href{https://www.mysql.com/}{https://www.mysql.com/}} as the most used SQL platform by developers on the sight\footnote{\href{https://insights.stackoverflow.com/survey/2017\#technology-databases}{https://insights.stackoverflow.com/survey/2017\#technology-databases}}. One reason for this is that some competitive SQL platforms require licensing (especially when commercial use is considered), such as Oracle SQL, while MySQL is open source and can be used more freely. The platform can be easily acquired and installed on various platforms (with major Windows and Linux operating systems supported), with largely straight forward initial configuration (for a project of this scale). The standard MySQL installation also includes the MySQL Workbench, which includes a convenient tool for building a relational database schema\footnote{\href{https://dev.mysql.com/doc/workbench/en/wb-data-modeling.html}{https://dev.mysql.com/doc/workbench/en/wb-data-modeling.html}}, with the ability to export SQL scripts to build the designed database in a single execution (including tables, relations and users with certain access rights). 

One of the most popular Java web frameworks available is Spring. Spring provides functionality to route requests to certain Java classes for processing, including the passing of any parameters with some degree of automatic validation, and the ability to process Java Servlet Pages (JSPs) which are compiled at run time to delivery varying pieces of information to the end users. It implemented the standard \textit{model-view-controller} (MVC) design pattern. To keep the focus on the use of data processed in this project, a convenient web container is desired, and that is what Spring Boot\footnote{\href{https://projects.spring.io/spring-boot/}{https://projects.spring.io/spring-boot/}} provides. This packages the Spring technology stack, web server technology (for example, Apache Tomcat) and any extra libraries required into a single, executable, Java project, making heavy use of annotations for the minimal configuration required. This makes it an extremely portable and (in terms of environment and system configuration) light weight application to set-up and run, which reflects as part of their projects mission statement to get developers "up and running as quickly as possible". It is a young project, with the original version being released in 2014 and version 2.0 being released during the course of this project, meaning it is built to work with the current standard of industrial Java technologies - so while this project is aiming to build a simple application, it shall have some powerful technologies standing behind it.

% TODO find a reference for MVC?

One of these technologies is Hibernate\footnote{\href{http://hibernate.org/}{http://hibernate.org/}}. This integrates with the Java Persistance API (JPA), supported by Spring, for extremely simply database interactions with many functions being supported by default. It combines a simple Java object which holds fields relating to each of the column in a database entity (table), and an abstract \texttt{CrudRepository} class which can be extended from to make a data access object which can retrieve records from a database and update them as well. Methods can be added to the extended class to make more complex queries, and either plain English can be used in the method name to define the query, or the specific SQL query statement can be specified if required\footnote{A guide to using this technology is freely available on Spring's website and is presented in a very simple manor. It can be viewed at \href{https://spring.io/guides/gs/accessing-data-jpa/}{https://spring.io/guides/gs/accessing-data-jpa/}} Furthermore, when fields are foreign keys of other entities, rather than being of an \textit{ID} type such as \texttt{Integer} or \texttt{Long}, the class representation of the foreign entity can be used instead, and Hibernate will automatically populate it with the correct object.

\section{Design and Configuration}

\subsection*{Design of the Services}
There are four key requirements of the POC system which cover a large amount of usability. The first requirement is very simple. Two pages are required for this:
\begin{itemize}
	\item The first is simply a list of papers, each selectable. When selected, the user will be taken to the next paper described.
	\item The second page, for displaying information about a paper, should begin with the title at the top, a section in the middle for the original text, and finish with a list of key phrases and relations. The key phrases, as a minimum, should be annotated on the main text as well.
\end{itemize}

The second requirement is covering the question of search. A new algorithm is proposed that includes the key phrase information about a paper. Using the concept of using the TF-IDF scores of both the tokens in all documents as well as the tokens in the query to try to find papers is utilised with a scaling factor, defined by the presence of key words in a paper.

\subsection*{Database Design and Entity Relationship Diagram}

\begin{figure}
	\includegraphics[width=\textwidth]{img/fypdberd.png}
	\caption[Database Entity Relationship Diagram]{The entity relationship diagram that supports this proof-of-concept system, as exported from the MySQL Workbench tool kit. Primary keys are represented by gold keys, foreign keys are red diamonds, normal fields are blue diamonds and hollow blue diamonds indicate the field is nullable. Types for all fields in all entities are also shown. The Java representation of these entities share the same name in all cases, converted to camel case, aside from the entity titles which have \texttt{DAO} appended (standing for \textit{data access object}) as without this the class names would clash with those used in the NLP system, which would have been confusing to program with. As examples, the \texttt{paper} entity was \texttt{PaperDAO} in the Java project, but an \texttt{id} field would remain \texttt{id} in the Java code to follow the standard.}
	\label{figure:dberd}
\end{figure}

The support storage and query of the data produced by the ScienceIE task, a suitable database must be designed and created. Given a lot of the required entities had been created as part of the NLP system, the data base design heavily inherits the classes and fields from this. The full design for the database is shown in figure \ref{figure:dberd}

The \texttt{paper} and \texttt{key\_phrase} entities are largely the same as their equivalents in the NLP system. Of course, rather than having a list of key phrases held within the \texttt{paper} entity, the \texttt{key\_phrase} entity has a field to store a reference to its parent \texttt{paper}. The \texttt{paper} entity also has two fields no in the NLP system: a \texttt{status} which was discussed above, and a \texttt{parse} which holds a serialised \texttt{Paper} Java object. This allows the preprocessing to take place and for that information to be stored directly in the database with its parent record, removing the need for re-computation of preprocessing every time the paper is called from the database.

The method for storing hyponyms and synonyms is slightly different to the NLP system. Rather than being part of a relations list the parent paper record holds, instead each relation holds the \texttt{key\_phrase} references they are a relation of. The original \texttt{paper} can still be retrieved through retrieving the referenced \texttt{key\_phrase} records. The \texttt{hyponym} record holds the two \texttt{key\_phrase} records it is a relation of, as well as a relative ID, which is the ID of the hyponym relation relative to the individual paper it is from. The \texttt{synonym} relation is a little more complex. While the NLP system created didn't support more than two way synonym relation extraction and there was only one example of a three way synonym relation in the ScienceIE data set, to support the range of relations defined as part of ScienceIE these synonym entity has to support one synonym referencing a variable amount of \texttt{key\_phrase} records. Therefore, to avoid a many-to-many relation between entities, each \texttt{synonym} record holds a reference to a \texttt{syn\_link}, the concept being a set of synonyms, each referencing one \texttt{key\_phrase}, all reference the same \texttt{syn\_link} record, which makes the synonym relation between the set of referenced key phrases. Having \texttt{syn\_link} also provides a convenient way to count he number of synonyms in the system.

\section{Implementation}

\subsection*{Project Configuration}
The creation of the project involved including various dependencies in the \texttt{pom.xml} file and setting up the launch configuration for Spring Boot.

To include Spring Boot, \texttt{pom.xml} was configured as described on the Spring Boot website, along with a connector package for MySQL\footnote{\href{https://spring.io/guides/gs/accessing-data-mysql/}{https://spring.io/guides/gs/accessing-data-mysql/}}. 

\begin{figure}
	\begin{lstlisting}[language=XML]
<dependency>
  <groupId>xyz.tomclarke.fyp</groupId>
  <artifactId>fyp-nlp</artifactId>
  <version>0.0.1-SNAPSHOT</version>
  <!-- Stop logging dependency errors -->
  <exclusions>
    <exclusion>
      <groupId>ch.qos.logback</groupId>
      <artifactId>logback-core</artifactId>
    </exclusion>
    <exclusion>
      <groupId>org.slf4j</groupId>
      <artifactId>slf4j-log4j12</artifactId>
    </exclusion>
  </exclusions>
</dependency>
	\end{lstlisting}
	\caption[Configuration to set the NLP system as a dependency in Maven]{The Maven configuration for listing the NLP project as a dependency, as used in the POC system. Note, the exclusion (as discussed) are to remove logging dependency conflicts using Spring Boot causes; this may break logging if another system uses this configuration but doesn't have its own logger available.}
	\label{figure:nlpdependency}
\end{figure}

To use the NLP system built in this project, Maven also needed to be told to import this so it can be used to generate information about given papers when the full system has been developed. Given the NLP system has been compiled and installed to the local Maven repository (as it is not hosted anywhere), the configuration shown in figure \ref{figure:nlpdependency} will include the NLP system in a compiled project. Doing this also sets any dependencies of the NLP system as dependencies of this system, so they do not need to be re-listed as dependencies. The unfortunate effect of this is where there are conflicts between dependencies, which is what happened in this project. The logging included in Spring Boot was an alternate version to that in the NLP project, and while it didn't damage any part of the system, on every boot it would choose one of the loggers and print out many error messages warning the developer against it. Therefore, the conflicting dependencies in the NLP system were excluded to remove this problem, and this was done to the NLP system rather than to Spring Boot as this problem may occur if the NLP system was used in another project including other dependencies, so it made sense to list how to fix it for the NLP system dependency. 

For the system to launch, some basic configuration had to be set. There is an \texttt{application.properties} file in the Java resources with the following parameters (with explanations as to why those were chosen appended):
\begin{itemize}
	\item \texttt{server.port=8080} 8080 is a standard port for testing web applications on. Furthermore, port 80 needs administrative privileges to be used ever time the application is launched, which would not only be frustrating to a developer, but not supporting good security standards if there was a vulnerability in the web service. The router the developer system was sitting behind, through port forwarding, allowed incoming connections on port 80 to be directed to 8080 on the developer system, so when connecting via web browser, no port would be needed to be specified.
	\item \texttt{debug=false} If \texttt{true} was set, this would output all debug information to the console. Given the log4j configuration being used output all debug information to disk, there was no need to have it also in console, making it harder to see what was going on while running.
	\item \texttt{spring.jpa.hibernate.ddl-auto=none} This parameter has several different possible values. \texttt{none} means the connection to the database is standard, allowing the Java application to commit reads and writes. Other values were possible, which would support (if the database hasn't been setup) automatic configuration of the database based on the entity declarations. While this could have been useful, control of configuring the database being left to the entity relationship software included in the MySQL workbench seemed preferable.
	\item \texttt{spring.datasource.url=\newline jdbc:mysql://tomclarke.xyz:3306/fyp?verifyServerCertificate=false\&useSSL=true} This is the JDBC connection string, allowing the system to find and connect to the database. To increase security, encryption of the connection through Secure Sockets Layer (SSL) is enabled, but as the database server is not configured with any certificate, this is set to be ignored.
	\item \texttt{spring.datasource.username=fyp\_user} This is simply the database user setup for the system to use. The permissions on this user were restricted to read and write of data in the \texttt{fyp} database, so if the application is compromised the database connection can't be used to read information from other databases on the same instance, or change the configuration of the database system. 
	\item \texttt{spring.datasource.password=$<$password$>$} This is the password of the database user for the system to login with.
\end{itemize}

Several parts of the above are concerned with, as well as required configuration, security of the database and system as whole, and while this project is not concerned with security, it is seen as good practise to include these features.

Finally, the class \texttt{FypGuiApplication} needed to be created, as it is this class that contains the classic \texttt{public static void main(String[] args)} method that starts the whole program. Along side the opportunity to include any other pieces of configuration completed through Java code, this initiated the Spring MVC technology stack.

\section{Web Interface}

\section{Testing}

\section{Conclusion}

\pagebreak
\chapter{Conclusion}
Overall, the solutions presented to the ScienceIE problem do not succeed in beating the solutions proposed as part of the task, however, do explore other idea about how to achieve them. Word2Vec usage in particular was explored and utilised through the solutions provided to all three subtasks. A more suitable model for Word2Vec would have been useful and likely boosted the performance of the systems. The reason for creating these systems is to extract classified key pieces of information from scientific papers, and a use for this was proposed and demonstrated in the proof-of-concept website produced. This used key phrase information when completing search to try to prioritise relevant papers to a users query, which was reasonably fast and accurate, given the data available. 

\pagebreak

%----------------------------------------------------------------------
% References and Appendicies

\pagebreak
\bibliographystyle{agsm}
\bibliography{fyp.bib}

\pagebreak
\begin{appendices}
\appendix

\chapter{Example ScienceIE Training/Test Document}
\label{appendix:egpaper}
The following is ScienceIE test paper file S0010938X15301268.txt:\\

\noindent Fig. 9 displays the growth of two of the main corrosion products that develop or form on the surface of Cu40Zn with time, hydrozincite (Fig. 9a) and Cu2O (Fig. 9b). It should be remembered that both phases were present already from start of the exposure. The data is presented in absorbance units and allows comparisons to be made of the amounts of each species between the two Cu40Zn surfaces investigated, DP and HZ7. The tendency is very clear that the formation rates of both hydrozincite and cuprite are quite suppressed for Cu40Zn with preformed hydrozincite (HZ7) compared to the diamond polished surface (DP). In summary, without being able to consider the formation of simonkolleite, it can be concluded that an increased surface coverage of hydrozincite reduces the initial spreading ability of the NaCl-containing droplets and thereby lowers the overall formation rate of hydrozincite and cuprite.

\chapter{Example ScienceIE Training/Test Annotation Data}
\label{appendix:egann}
The following is ScienceIE test paper annotations file S0010938X15301268.ann:\\

\noindent T1	Material 46 64	corrosion products
T2	Material 104 110	Cu40Zn\\
T3	Material 122 134	hydrozincite\\
T4	Material 149 153	Cu2O\\
T5	Material 378 384	Cu40Zn\\
T6	Material 408 410	DP\\
T7	Material 415 418	HZ7\\
T8	Material 530 536	Cu40Zn\\
T9	Material 552 564	hydrozincite\\
T10	Material 566 569	HZ7\\
*	Synonym-of T9 T10\\
T11	Material 587 611	diamond polished surface\\
T12	Material 613 615	DP\\
*	Synonym-of T11 T12\\
T13	Material 678 691	simonkolleite\\
T14	Material 751 763	hydrozincite\\
T15	Material 809 833	NaCl-containing droplets\\
T16	Material 883 895	hydrozincite\\
T17	Material 900 907	cuprite\\
T18	Process 456 471	formation rates\\
T20	Process 280 296	absorbance units\\
T19	Task 308 406	comparisons to be made of the amounts of each species between the two Cu40Zn surfaces investigated\\
R1	Hyponym-of Arg1:T3 Arg2:T1\\
R2	Hyponym-of Arg1:T4 Arg2:T1\\
T21	Material 480 492	hydrozincite\\
T22	Material 497 504	cuprite\\
T23	Process 665 691	formation of simonkolleite\\
T24	Process 776 793	initial spreading\\
T25	Process 865 879	formation rate\\

\chapter{Stop Words List}
\label{appendix:stopwords}
Below is the list of stop words used in this project:\\
\begin{multicols}{7}
\noindent !!\\
?!\\
??\\
!?\\
`\\
``\\
''\\
-lrb-\\
-rrb-\\
-lsb-\\
-rsb-\\
,\\
.\\
:\\
;\\
"\\
'\\
?\\
$<$\\
$>$\\
\{\\
\}\\
$[$\\
$]$\\
+\\
-\\
(\\
)\\
\&\\
\%\\
\$\\
@\\
!\\
\^\\
\#\\
*\\
..\\
...\\
'll\\
's\\
'm\\
a\\
about\\
above\\
after\\
again\\
against\\
all\\
am\\
an\\
and\\
any\\
are\\
aren't\\
as\\
at\\
be\\
because\\
been\\
before\\
being\\
below\\
between\\
both\\
but\\
by\\
can\\
can't\\
cannot\\
could\\
couldn't\\
did\\
didn't\\
do\\
does\\
doesn't\\
doing\\
don't\\
down\\
during\\
each\\
few\\
for\\
from\\
further\\
had\\
hadn't\\
has\\
hasn't\\
have\\
haven't\\
having\\
he\\
he'd\\
he'll\\
he's\\
her\\
here\\
here's\\
hers\\
herself\\
him\\
himself\\
his\\
how\\
how's\\
i\\
i'd\\
i'llv
i'm\\
i've\\
if\\
in\\
into\\
is\\
isn't\\
it\\
it's\\
its\\
itself\\
let's\\
me\\
more\\
most\\
mustn't\\
my\\
myself\\
no\\
nor\\
not\\
of\\
off\\
on\\
once\\
only\\
or\\
other\\
ought\\
our\\
ours\\
ourselves\\
out\\
over\\
own\\
same\\
shan't\\
she\\
she'd\\
she'll\\
she's\\
should\\
shouldn't\\
so\\
some\\
such\\
than\\
that\\
that's\\
the\\
their\\
theirs\\
them\\
themselves\\
then\\
there\\
there's\\
these\\
they\\
they'd\\
they'll\\
they're\\
they've\\
this\\
those\\
through\\
to\\
too\\
under\\
until\\
up\\
very\\
was\\
wasn't\\
we\\
we'd\\
we'll\\
we're\\
we've\\
were\\
weren't\\
what\\
what's\\
when\\
when's\\
where\\
where's\\
which\\
while\\
who\\
who's\\
whom\\
why\\
why's\\
with\\
won't\\
would\\
wouldn't\\
you\\
you'd\\
you'll\\
you're\\
you've\\
your\\
yours\\
yourself\\
yourselves\\
\#\#\#\\
return\\
arent\\
cant\\
couldnt\\
didnt\\
doesnt\\
dont\\
hadnt\\
hasnt\\
havent\\
hes\\
heres\\
hows\\
im\\
isnt\\
its\\
lets\\
mustnt\\
shant\\
shes\\
shouldnt\\
thats\\
theres\\
theyll\\
theyre\\
theyve\\
wasnt\\
were\\
werent\\
whats\\
whens\\
wheres\\
whos\\
whys\\
wont\\
wouldnt\\
youd\\
youll\\
youre\\
youve\\
\end{multicols}

\chapter{Google Scholar and ScienceDirect Search Pages}
\label{appendix:searcheg}
The following are screen shots of the query "computer science" being queried in two popular search engines: Google Scholar (left) and ScienceDirect (right). The "\textbf{$\cdot$$\cdot$$\cdot$}" denote some of the search results has been cropped out. As cropping has been applied, the results presented by Google Scholar and ScienceDirect are 10 and 25 (configurable to 50 or 100) respectively. \\

\includegraphics[width=\textwidth]{img/searchexamples.png}

\chapter{How to run the project from source}
\label{appendix:howtorun}

To run the two applications, the following prerequisite steps must be taken:
\begin{enumerate}
	\item Install Java 8, Maven 3 and MySQL.
	\begin{itemize}
		\item Running \texttt{./install\_*.sh} in \texttt{resources/scripts} will do this for a system with \texttt{apt-get} available, such as Ubuntu.
	\end{itemize}
	\item Run the \texttt{env\_maven\_opts.sh} script in \texttt{resources/scripts} to setup required system environment variables.
	\item A Word2Vec model is needed. Based on this report, the pre-trained Google News model\footnote{\href{https://drive.google.com/file/d/0B7XkCwpI5KDYNlNUTTlSS21pQmM/}{https://drive.google.com/file/d/0B7XkCwpI5KDYNlNUTTlSS21pQmM/}} is recommended for download.
\end{enumerate}

\noindent Some files required by the system need pointing to. 
In terms of training and testing paper locations, checking the \texttt{paper.txt} and \texttt{test\_papers.txt} in the resource folders of both Java projects should reveal the paths used on the development system. The paths listed in this file are all processed unless they begin with a \texttt{\#}, and either point to a text file for processing, or a directory containing multiple text files for processing. Change these appropriately. 

Other components are more embedded in the Java code. Ideally a better solution to this should have been included by the developer, but unfortunately other tasks used up this time. Change \texttt{log4j2.xml:8} and \texttt{NlpObjectStore.java:23} to any folder where logs and serialised files can be saved. Change \texttt{Word2VecPretrained.java:11} to point to a Word2Vec model (either the Google News model or a model renamed to the same file descriptor).

To setup the MySQL database, navigate to \texttt{resources/sql}. Run, in order, \texttt{./setup.sh} and \texttt{./build.sh}. If you wish to use the final data made in this project run \texttt{./recover.sh}. Depending upon your MySQL configuration, you may need to change the password to be stronger. Finally, update \texttt{application.properties} in the FYP-GUI project's resource folder with the correct connection string and password.

Navigate to the root of the two Java projects (both in \texttt{java/}), and a series of useful scripts for compiling and running are presented: \\

\noindent \begin{tabular}{ l | p{3.5cm} | p{9cm} }
	\textbf{Java Project} & \textbf{Script} & \textbf{Description} \\
	\hline
	FYP-NLP & \texttt{./build.sh} & Compiles the system, without running any tests. \\
	\hline
	FYP-NLP & \texttt{./install.sh} & Compiles the system, without running any tests, and installs the project to the local Maven repository (required for building the FYP-GUI project). \\
	\hline
	FYP-NLP & \texttt{./test.sh $<$test class$>$} & Compiles the system and runs the JUnit test class given. Test classes can be found in \texttt{java/FYP-NLP/src/test/java/xyz/tomclarke/fyp/nlp}, and specifying the name of the class will run all test methods inside it without \texttt{@Ignore} above the \texttt{@Test} annotation.  \\
	\hline
	FYP-GUI & \texttt{./build.sh} & Compiles the GUI and runs all JUnit tests. \\
	\hline
	FYP-GUI & \texttt{./build-and-run.sh} & Compiles the GUI, without running any tests, and launches the GUI. \\
	\hline
	FYP-GUI & \texttt{./run.sh} & Launches the GUI (assumes it is already built). \\
\end{tabular}

\end{appendices}

\end{document}
