\chapter{Project Architecture}
With any large software project, it is sensible to choose a platform with all the necessary tools available so the developer can achieve their goals. The following describes the environment and technologies used generically through out the entire project.

\section{Language}
Due to the past experience of the author, Java was an obvious choice. Given extensive time working in the language during university and in industry, a thorough understanding of the programming language was already achieved, which allowed for planning of a sensible software architecture to optimise code quality and (implicitly) the potential of increased success of the systems created. 

Furthermore, Java is a very popular and accessible language world wide - backed up by the active StackOverflow community (casual and professional alike) with Java being one of the most popular technologies for at least the last five years, evidenced through their user surveys 2018\footnote{\href{https://insights.stackoverflow.com/survey/2018}{https://insights.stackoverflow.com/survey/2018}}, 2017\footnote{\href{https://insights.stackoverflow.com/survey/2017}{https://insights.stackoverflow.com/survey/2017}} and 2016\footnote{\href{https://insights.stackoverflow.com/survey/2016}{https://insights.stackoverflow.com/survey/2016}}. Due to this, Java has extensive support for many common problems people encounter, with issues being discussed and solutions proved across various forums. 

Not only is Java's popularity good for increasing support availability, many libraries and utilities are available to help developers with tasks. Along side other technologies used for more specific tasks throughout completion of the NLP system and the POC system (which shall be discussed when used), common technologies used during the development of the entire project are described below. Throughout development of the project, very little issue was caused by lack of Java support for common processes or lack of Java capability when attempting to program some process (which was a critical part of evaluating which language should be used). 

Finally, Java serialisation was used throughout (and will be noted when is). Serialisation allows the system to save a Java object to disk (any file name can be chosen, but classically its postfix is \texttt{.ser}), and later be reloaded. 

As a brief aside, Python is another extremely popular language used for NLP and likely could have been used for at least the first half of this project producing similar results.

\subsection*{log4j}
log4j 2\footnote{\href{https://logging.apache.org/log4j/2.x/}{https://logging.apache.org/log4j/2.x/}} is a popular and robust library developed under the Apache Software Foundation to do logging in Java. It's useful features include:
\begin{itemize}
	\item Automatic output of logs to both terminal and file: As well as immediate visual feedback, log files can be used for later processing and evidence gathering.
	\item Timing of events: Timing is very useful as during long runs of a system (for example, some sections of the NLP task could take hours to complete) the logs can be analysed to see how long systems take to process data, which can be considered when going forward; for instance in terms of formulating efficiently timed tests plans.
	\item Labelling of logs into levels such as \textit{debug}, \textit{info}, \textit{error} and \textit{fatal} messages: This can be used when analysing the logs to catch where things went wrong (filtering for error messages) and then to try to debug the system by finding information logged prior to that (with debug). During development an excellent use of this feature is to output all levels aside from 'debug' to terminal, so monitoring progress isn't overloading the executor with information, but if something does go awry the steps leading up to the bad event can be analysed in the log saved to disk.
	\item Specified layout of logs: the developer of a project can detail what information (and the precision of the information) is included in a log statement (for example, time of the log, source of the log). This, along with all other configuration for using log4j 2, is completed in \texttt{log4j2.xml} in a Java projects \texttt{resource} folder.
\end{itemize}

While direct output of this will not be present in the rest of this report, it is worth noting this was an extremely useful tool for developing all of the systems to follow. 

\subsection*{Maven}
Apache Maven\footnote{\href{https://maven.apache.org/}{https://maven.apache.org/}} is another important tool. Like log4j, it is developed by the Apache foundation. 

Maven is a tool to help with project management and has many uses. It is based around a \textit{project object model} (POM) configured in a \texttt{pom.xml} file at the root of a Java project, which itself has a structure defined by Maven. The key uses utilised in this project are:
\begin{itemize}
	\item Project compilation: Maven can be used to build a project and automatically run specified or all tests, with more detailed and well formatted output than compiling Java code by hand. Therefore, compilation and testing can more easily be scripted and output more clearly analysed. It also handles importing libraries used in a Java project when compiling (which can be very troublesome when completed by hand), which is discussed below.
	\item Library import: The \texttt{pom.xml} can specify dependencies of the Java project. While custom, third party repositories exist, Maven has a central repository\footnote{\href{http://repo.maven.apache.org/maven2/}{http://repo.maven.apache.org/maven2/}} with many libraries available. This includes log4j described above, and all other libraries used in this project. Dependencies are downloaded to the systems local Maven repository at compile time.
	\item Library export: As discussed in the introduction, the NLP system shall be used in a POC system. Rather than combining these two systems into one large package, or doing a confusing copy of the required resources, Maven can be used to export the compiled NLP system to the local Maven repository. Then, the POC system can simply list the NLP system as a dependency, and Maven shall include it as a library when building the executable program.
\end{itemize}

Maven is used as the management backbone throughout the development of software discussed in this report. When libraries are used in a project, a link to their dependency configuration for Maven's \texttt{pom.xml} shall be included. As a good example, log4j\footnote{\href{https://logging.apache.org/log4j/2.x/maven-artifacts.html}{https://logging.apache.org/log4j/2.x/maven-artifacts.html}} has an extensive page providing a detailed description of how to import the library.

\subsection*{JUnit}
JUnit is a popular Java framework for testing. It is simple to use, catching unexpected (or expected) exceptions and ensuring values are correct with \texttt{assert} statements. 

Maven also integrates with it, so that (by default) when you build a Java project with Maven, all of the methods marked with \texttt{@Test} annotation in the test source directory are executed to ensure the program is working as expected (as far as the tests ensure that). It will then provide a report and trace of any issues once complete. Maven will also automatically exclude the test files from the final packaged product to reduce waste space for deployments of projects. 

While working through this project many JUnit tests were constructed (all of which are still available in the Git repository for this project). Somewhat unconventionally, there is a divide between tests: while some are based around ensuring functionality works as expected, many are actually building the NLP systems, training them (if required), testing them and comparing the predictions made to the gold standard data.

The tests can also be ignored\footnote{\href{http://maven.apache.org/surefire/maven-surefire-plugin/examples/skipping-tests.html}{http://maven.apache.org/surefire/maven-surefire-plugin/examples/skipping-tests.html}} which is very useful, as many of the tests written are base around evaluating the algorithms created rather than testing functionality; so not only does not every algorithm need to be retested at every compilation time, but if they were it would take many hours (and probably more memory than the standard computer has) to build and test the application. 

\subsection*{Word2Vec}
The interesting Word2Vec technology is utilised in this project in various places. The original Word2Vec library implementation was in Python. However, the Deep Learning For Java (DL4J) team have included, as part of their machine learning and deep neural network library, Word2Vec funcitonality\footnote{\href{https://deeplearning4j.org/word2vec.html}{https://deeplearning4j.org/word2vec.html}}. This supports training a Word2Vec model, using the model, and saving and loading models. 

The models used in this project are the Google News model\footnote{\href{https://drive.google.com/file/d/0B7XkCwpI5KDYNlNUTTlSS21pQmM/edit?usp=sharing}{https://drive.google.com/file/d/0B7XkCwpI5KDYNlNUTTlSS21pQmM/edit?usp=sharing}} and the Freebase model\footnote{\href{https://docs.google.com/file/d/0B7XkCwpI5KDYaDBDQm1tZGNDRHc/edit?usp=sharing}{https://docs.google.com/file/d/0B7XkCwpI5KDYaDBDQm1tZGNDRHc/edit?usp=sharing}}. While neither of these are made up of scientific articles, they both have a large vocabulary size (3 million and 1.4 million tokens respectively), and both based off of a 100 GB large samples, which should allow them to perform relatively well. This is the reason Word2Vec is being used over GloVe, as these models have much larger vocabulary sizes that the pre-trained GloVe models found, so in theory have a better change of covering the vocabulary of the ScienceIE data set\footnote{This GitHub page contains a selection of popular models for both Word2Vec and GloVe: \href{https://github.com/3Top/word2vec-api}{https://github.com/3Top/word2vec-api}}.

Attempts were made to use a Wikipedia based model (Wiki2Vec\footnote{\href{https://github.com/idio/wiki2vec}{https://github.com/idio/wiki2vec}}) but unfortunately no successful attempt was made to use it in this project (there were various problems converting the model to a Java readable format and loading it). While potentially of lower quality semantics (as Wikipedia isn't officially maintained) it may have had more of the vocabulary the ScienceIE data supports as Wikipedia covers many topics including those of scientific nature so could have increased coverage of the model when finding similarities between various scientific tokens. 

\section{Platform}
While Java is cross platform (another excellent reason for using it), some of the underlying system libraries that Wor2Vec relies on to function are included by default in many Linux distributions. It can be made to work on Microsoft Windows operating systems, but it requires a large amount of complex configuration and generally not worth the pay off. Therefore, this system was built on the Ubuntu 16.04 distribution of Linux, as this involved the least amount of configuration to get working.

Furthermore, some of the algorithms created as part of this paper are able to fill up available memory on a computer very quickly. The memory available as part of this project was 16 GB. Linux swap space was configured (an \textit{overflow} area for memory usage) but generally one would not like to use this, as it is slow to read from and will add some wear to the solid state drive in the host system available (due to many fast reads and writes) which isn't good. This is another reason for using Linux as the platform to build these systems on, as the memory overhead from the operating system is much smaller when compared with Microsoft Windows 10 (in the order of gigabytes of memory saved). Furthermore, Linux can also be run headless (without a GUI) to further reduce the operating systems memory usage, which on Ubuntu 16.04 saves approximately an extra gigabyte of memory. With support for Secure Shell (SSH) to remotely connect to the system, to run tests and read results, Linux is an excellent choice of platform for optimising memory usage while running these algorithms.
